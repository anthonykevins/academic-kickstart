\documentclass[letterpaper]{scrartcl}
\usepackage{tabularx}
\usepackage[none]{hyphenat}
\usepackage[utf8x]{inputenc}
\usepackage{lmodern,textcomp}
\usepackage{enumitem}
\usepackage{ltablex}
\usepackage[english]{babel}
\usepackage{blindtext}
\usepackage{microtype}
\usepackage[hidelinks]{hyperref}
\usepackage{hanging}
\usepackage[paperheight=279.4mm,paperwidth=210mm,bottom=10em]{geometry}
\usepackage{fancyhdr}
\setlength{\parskip}{1em}

\pagestyle{fancy}
\fancyhf{}
\fancyhead[LE,LO]{Anthony Kevins}
\fancyhead[RE,RO]{Research Plan}
\fancyfoot[CE,CO]{\leftmark}
\fancyfoot[CE,CO]{\thepage}
\setlength\parindent{0pt}
\setlist[itemize]{leftmargin=*}

\begin{document}

  \raggedbottom

  Who deserves our collective support? This is a key question in contemporary democracies, and answers to it have varied dramatically across time, countries, and policy domains. On the one hand, social solidarity and generous social programmes seem to have existed in a virtuous circle of mutual reinforcement. Yet, that solidarity has always obscured some degree of ``us'' versus ``them'' distinctions -- often excluding immigrants, the ``non-deserving'' poor, or certain ethnic minorities -- and this tendency appears to have accentuated in recent years. Motivated by these tensions, my research explores the relationship between public policy, inequality, and the division between those who we deem worthy of solidarity and those who we exclude from it.

  My research plan centres around three questions that are fundamental to these discussions: What institutional factors shape relations between societal insiders and outsiders? What determinants drive the nature and timing of welfare state reform? And how can we best understand social policy preferences? In what follows, I lay out my research agenda for the coming years through a discussion of these questions, referencing both ongoing and future work on these themes. Taken together, these projects will allow me to make a meaningful contribution to debates in Comparative Politics and Political Behaviour.

  \textbf{(1) Societal Insiders and Outsiders}
  \vspace{-1em}

  First, my research aims to improve our understanding of how institutions affect the divide between groups that have historically benefited from a privileged access to state protection (i.e. insiders) and those that have typically been excluded from it (i.e. outsiders).

  In doing so, I build on both the broader dualization literature and several of my recently published studies, which include: a (peer-reviewed) book examining when and why governments provide welfare state outsiders with access to healthcare and unemployment benefits (Kevins, 2017); an article investigating how labour market vulnerability interacts with active labour market policy expenditure to shape generalized trust (Kevins, 2018); and a paper that leverages the contrasting fates of French and Italian minimum income schemes to draw out the determinants of access to social assistance (Kevins, 2015).

  My ongoing projects on this topic can be split into two subthemes. The first, most closely related to my existing work, focuses on the causes and consequences of labour market precarity. One paper in this vein uses micro-level data from the European Union Statistics on Income and Living Conditions (EU-SILC) alongside data from the European Social Survey to examine the effects of labour market vulnerability on opinions about immigration (Kevins and Lightman, N.d.). Two other studies, in turn, concentrate directly on experiences in the labour market: the first explores the interactive effects of care work, professionalization, and inequality on job satisfaction (Lightman and Kevins, N.d.); while the second (still in development) will use EU-SILC microdata to measure labour market exploitation and analyze its correlates (Kevins and Paraciani, N.d.).

  The second subtheme, in turn, extends my investigation of welfare state and labour market divisions to the divide between democratic insiders and outsiders. This project, funded by a Marie Curie Individual Fellowship, uses survey experiments in the US (completed in summer 2018) and the Netherlands (to be conducted in early 2019) to study public opinion toward the influence of groups that are disproportionately impacted by a given policy decision (as, for example, with people of colour and policing policy). The project explores the circumstances under which citizens believe that policy-affected individuals should have more or less influence than others; to that end, I am currently investigating how Americans' reactions to consultation procedures are shaped by (a) racial resentment (Kevins and Robison, N.d.) and (b) the nature of individual policy preferences (Kevins, N.d.a.). Findings from this project will eventually also include a comparison of Dutch and American respondents as well as a co-authored piece (with Barbara Vis) exploring the impact of risk aversion on Dutch attitudes toward differentiated representation. Within a few years, I will then build upon this research theme by seeking funding for a project studying how economic vulnerability and racial resentment shape attitudes about how policies should be made and implemented.

  \textbf{(2) Welfare State Design and Expenditure}
  \vspace{-1em}

  Unpacking the relationship between poverty, inequality, and solidarity in advanced democracies clearly also requires a nuanced understanding of the welfare state. A second strand of my research is thus focused on social programme design and expenditure.

  In addition to my book, which contains several chapters on the bi-directional interplay between welfare state design and public opinion (Kevins, 2017), I have already published numerous co-authored studies on this topic. These publications include articles that: explore the relationship between welfare state universalism and immigrant integration (Kevins and van Kersbergen, Forthcoming); examine the impact of problem pressure on radical social policy innovation via a case study of late 19th-century Germany (Horn and Kevins, 2018); use corpus data from the Comparative Manifesto Project to push forward debates about how parties discuss the welfare state, equality, and social justice (Horn et al., 2017); and investigate the relationship between immigration and welfare state expenditure across various policy domains (Soroka et al., 2016).

  My current research, in turn, concentrates on the role of political parties in shaping the welfare state. One such paper examines the extent to which parties attempt to tie the welfare state to clientelistic appeals to specific social groups (Horn et al., N.d.a.); while another will explore the relationship between party families and workfare reforms -- that is, policy changes that increase the emphasis on conditions and obligations for unemployed benefit seekers (Horn et al., N.d.b.). The latter study is based off of an original dataset developed with Kees van Kersbergen and Alexander Horn. Specifically, we designed a new measure of workfare reform – the ‘workfare balance’ – that measures the rights and obligations of unemployment benefit recipients. We then worked with student assistants to code hundreds of laws and assemble a comprehensive and nuanced database on workfare policies across sixteen OECD countries from 1980 through to 2015. In addition to providing us with material for several papers, we believe that the dataset itself will also serve as a significant contribution to the literature.

  \textbf{(3) Public Opinion and Social Policy}
  \vspace{-1em}

  Third, my research seeks to develop our understanding of the public’s redistributive and social policy preferences, as well as the factors that influence them.

  My recent publications on this topic examine these attitudes from numerous perspectives, investigating: the relevance of class for one's stance toward the welfare state (Kevins et al., 2018a); the effects of different types of inequality on redistributive preferences across the income spectrum (Kevins et al., 2018b); long-term trends in partisan sorting vis-à-vis social policy preferences in Canada (Kevins and Soroka, 2018); and the impact of assertions about benefit generosity on support for social assistance in the United Kingdom (Jensen and Kevins, 2018).

  Some of this work is based off of an original survey on social policy preferences fielded in nine European countries and the US as part of a project on universalism and the welfare state (with Carsten Jensen, Kees van Kersbergen, and Alexander Horn). We also intend for two further articles to come out of this work: one investigating cross-national patterns in the motives citizens ascribe to others who are pro- or anti-redistribution (Kevins et al., N.d.b.); and another, more descriptive paper (to be submitted to \textit{European Political Science}) that will coincide with the public release of the data (Jensen et al., N.d.).

  Two final projects round out my work on public opinion and social policy. The first uses a new workfare dataset (described above) to contribute to the literature on the attitudinal consequences of welfare programme reform: by combining data on the workfare balance with public opinion data from the International Social Survey Programme, we will examine the relationship between workfare policies on social solidarity and support for the welfare state (Kevins et al., n.d.a.). A second, larger research project (currently in development) takes as its starting point the surprising resilience of popular support for the welfare state in public opinion data; together with a colleague at the University of Essex, I will seek funding for an original, cross-nationally fielded survey project designed to break down the nature of responses to the standard welfare state survey items (Kevins and Lee, N.d.). More generally, this latter project will also help to extend the comparative behavioural strand of my research programme, complementing both my Marie Curie project and several other projects in development (Kevins, N.d.b.; Lee and Kevins, N.d.).

  \textbf{Conclusion}
  \vspace{-1em}

  Overall, my research aims to expand our understanding of how parties, public opinion, and institutions interact to shape social solidarity, especially in the face of increasing inequality, labour market segmentation, and economic austerity. To that end, my published, ongoing, and planned studies engage with debates in Comparative Politics about the effects of institutions on insiders and outsiders, the factors shaping welfare state design and expenditure, and the nature and determinants of social policy preferences. It is my hope, however, that the results of this research will have meaningful real-world implications as well -- most importantly, helping us to better understand why government policy varyingly reduces, reinforces, and generates inequality.

  \pagebreak

  \textbf{References:}
  \vspace{-.5em}

  \begin{hangparas}{.25in}{1}

    Horn, Alexander, and Anthony Kevins. (2018) ``Problem Pressure and Social Policy Innovation: Lessons from 19th-Century Germany.'' \textit{Social Science History}, 42(3): 495-515.
    \vspace{-.5em}

    Horn, Alexander, Anthony Kevins, Carsten Jensen, and Kees van Kersbergen. (2017) ``Peeping at the Corpus: What is Really Going on Behind the Equality and Welfare Items of the Manifesto Project.'' \textit{Journal of European Social Policy}, 27(5): 403-416.
    \vspace{-.5em}

    Horn, Alexander, Anthony Kevins, Carsten Jensen, and Kees van Kersbergen. (N.d.a.) ``How Parties (Do Not) Appeal to Social Groups: Hollowing of Democracy or Hyper- Adaptive Targeting?''
    \vspace{-.5em}

    Horn, Alexander, Anthony Kevins, and Kees van Kersbergen. (N.d.b.) ``Partisanship and Workfare: Introducing and Applying a New Measure of Rights and Obligations in OECD Countries.''
    \vspace{-.5em}

    Jensen, Carsten, and Anthony Kevins. (2018) ``Numbers and Outrage in Welfare State Politics.'' OnlineFirst in \textit{Political Studies}.
    \vspace{-.5em}

    Jensen, Carsten, Kees van Kersbergen, Anthony Kevins, and Alexander Horn. (N.d.) ``Universalism and the Welfare State: A New Survey Dataset from ten Advanced Democracies.''
    \vspace{-.5em}

    Kevins, Anthony. (2015) ``Political Actors, Public Opinion, and the Extension of Welfare Coverage.'' \textit{Journal of European Social Policy}, 25(3): 303-315.
    \vspace{-.5em}

    Kevins, Anthony. (2017) \textit{Expanding Welfare in an Age of Austerity: Increasing Protection in an Unprotected World}. Amsterdam, NL: Amsterdam University Press.
    \vspace{-.5em}

    Kevins, Anthony. (2018) ``Dualized Trust: Risk, Social Trust, and the Welfare State.'' OnlineFirst in \textit{Socio-Economic Review}.
    \vspace{-.5em}

    Kevins, Anthony. (N.d.a.) ``Who Should have a Say?: Preferences for Differentiated Representation.''
    \vspace{-.5em}

    Kevins, Anthony. (N.d.b.) ``Race, Class, or Both?: Responses to Candidate Characteristics in the UK and Canada.''
    \vspace{-.5em}

    Kevins, Anthony, Alexander Horn, and Kees van Kersbergen. (N.d.a.) ``How Workfare Shapes Social Solidarity: Cross-National Trends in Workfare and Public Opinion Since the 1980s.''
    \vspace{-.5em}

    Kevins, Anthony, Alexander Horn, Carsten Jensen, and Kees van Kersbergen. (2018a) ``The Illusion of the Class in Welfare State Politics?'' OnlineFirst in the \textit{Journal of Social Policy}.
    \vspace{-.5em}

    Kevins, Anthony, Alexander Horn, Carsten Jensen, and Kees van Kersbergen. (2018b) ``Yardsticks of Inequality: Median Voter Preferences for Redistribution in Advanced Democracies.'' \textit{Journal of European Social Policy}, 28(4): 402-418.
    \vspace{-.5em}

    Kevins, Anthony, Alexander Horn, Carsten Jensen, and Kees van Kersbergen. (N.d.b.) ``Motive Attribution, Redistribution, and the Moral Politics of the Welfare State.''
    \vspace{-.5em}

    Kevins, Anthony, and Seonghui Lee. (N.d.) ``Decomposing Public Support for the Welfare State.''
    \vspace{-.5em}

    Kevins, Anthony, and Naomi Lightman. (N.d.) ``Immigrant Sentiment and Labour Market Vulnerability: Economic Perceptions of Immigration in Dualised Labour Markets.''
    \vspace{-.5em}

    Kevins, Anthony, and Rebecca Paraciani. (N.d.) ``Causes and Consequences of Labour Market Exploitation.''
    \vspace{-.5em}

    Kevins, Anthony, and Joshua Robison. (N.d.) ``Race and Differentiated Representation: The Case of Police Guidelines.'' \vspace{-.5em}

    Kevins, Anthony, and Stuart N. Soroka. (2018) ``Growing Apart? Partisan Sorting in Canada, 1992-2015.'' \textit{Canadian Journal of Political Science}, 51(1): 103-133.
    \vspace{-.5em}

    Kevins, Anthony, and Kees van Kersbergen. (Forthcoming) ``The Effects of Welfare State Universalism on Migrant Integration.'', \textit{Policy \& Politics}. \vspace{-.5em}

    Lee, Seonghui, and Anthony Kevins. (N.d.) ``No Party, No Problem: Inferring Partisanship from Label-less Candidate Vignettes.'' \vspace{-.5em}

    Lightman, Naomi, and Anthony Kevins. (N.d.) ``Bonus or Burden?: Care Work and Job Satisfaction in Eighteen European Countries.''
    \vspace{-.5em}

    Soroka, Stuart N., Richard Johnston, Anthony Kevins, Keith Banting, and Will Kymlicka. (2016) ``Migration and Welfare State Spending.'' \textit{European Political Science Review}, 8(2): 174-193.

  \end{hangparas}

\end{document}
