\documentclass[fontsize=10.5pt,letterpaper]{scrartcl}
\usepackage{tabularx}
\usepackage[none]{hyphenat}
\usepackage[utf8x]{inputenc}
\usepackage{enumitem}
\usepackage{lmodern,textcomp}
\usepackage{ltablex}
\usepackage[english]{babel}
\usepackage{blindtext}
\usepackage{microtype}
\usepackage{hanging}
\usepackage[paperheight=279.4mm,paperwidth=210mm,bottom=10em,footskip=0.4in]{geometry}
\usepackage{fancyhdr}
\setlength{\parskip}{1em}
\setlength{\headsep}{0.2in}
\renewcommand{\footnotesize}{\scriptsize}
\usepackage[hang,flushmargin]{footmisc}
\usepackage[hidelinks]{hyperref}

\pagestyle{fancy}
\fancyhf{}
\fancyhead[LE,RO]{Anthony Kevins}
\fancyhead[RE,LO]{NRF Application}
\fancyfoot[CE,CO]{\leftmark}
\fancyfoot[LE,CO]{\thepage}
\setlength\parindent{0pt}
\setlist[itemize]{leftmargin=*}

\begin{document}

  \raggedbottom

{\large \textbf{Economic Anxiety, Racial Resentment, and Democratic Preferences}}

\textbf{Project Nature and Aims}
\vspace{-.9em}

How do economic vulnerability and racial resentment shape ideas about the way democracy should work? The increasing attraction of populist parties and politicians across the West has triggered debate about the role of economic and racial anxieties in fomenting populism.\footnote{E.g. Mutz, D. C. (2018). Status threat, not economic hardship, explains the 2016 presidential vote. \textit{Proceedings of the National Academy of Sciences}, OnlineFirst.} My project aims to extend our understanding of these relationships by studying how economic vulnerability and racial resentment impact \textit{process preferences} – that is, ideas about how policies should be made and implemented. Studying these process preferences alongside vote choice and abstract populist attitudes will, in turn, enhance our comprehension of the dynamics underlying populism’s growing appeal and its practical significance for democratic governance. By unpacking these links, my project will have important implications for all those seeking to better understand trends in policymaking, policy implementation, and government composition.

This project starts from a simple, yet often ignored paradox underlying the relationship between populism and liberal democracy: although populism’s focus on the will of “the people” puts it in tension with liberal democracy, “populism is in many ways an illiberal democratic response to undemocratic liberalism”.\footnote{Mudde, C., \& Rovira Kaltwasser, C. (2018). Studying populism in comparative perspective: reflections on the contemporary and future research agenda. \textit{Comparative Political Studies}, OnlineFirst, p. 4.} Populism’s practical implications for democracy are thus far from clear, especially if we assume that populists are interested in taking up the democratic grievances of voters. Further complicating this picture, economic and racial anxieties likely play a dual role in shaping ideas about how democratic decisions should be taken: by nourishing populist inclinations, these anxieties may contribute to a more citizen-centred conception of democracy; yet, by underlining economic and racial tensions within the citizenry, they may also reduce the desired influence of certain types of citizens. To better grasp the nuances of these dynamics, we must therefore investigate their impact on process preferences, focusing on the desired roles of politicians, pressure groups, and stakeholders – in particular where diverse economic and racial groups are involved.

In approaching this topic, the project builds from two strands of political science research. First, various studies have examined the determinants of support for populist parties and politicians, in the process demonstrating the importance of economic and racial anxieties. On the one hand, research on the effects of precarity has highlighted that increased economic risk exposure likely has negative democratic consequences, including greater political alienation, lower external political efficacy, and an increased propensity to vote for anti-system parties.\footnote{E.g. Meyer, B. (2018). Left to right: labour market policy, labour market status and political affinities. \textit{Journal of Public Policy}, OnlineFirst.} On the other, a growing number of articles suggest that racial anxiety is actually the much more important determinant of vote choice;\footnote{E.g. Sides, J., Tesler, M., \& Vavreck, L. (2018). Hunting where the ducks are: activating support for Donald Trump in the 2016 Republican primary. \textit{Journal of Elections, Public Opinion and Parties}, 28(2), 135-156.} but given that these studies have tended to measure economic vulnerability in comparatively simplistic terms, it is hard to draw any firm conclusions about relative impacts.

A separate body of scholarship is in turn interested in the democratic preferences of citizens. While research on process preferences looks at the causes and consequences of specific types of decision-making procedures,\footnote{E.g. Arnesen, S., \& Peters, Y. (2018). The legitimacy of representation: how descriptive, formal, and responsiveness representation affect the acceptability of political decisions. \textit{Comparative Political Studies}, 51(7), 868-899.} another, distinctly populist-oriented strand of the literature focuses on abstract ideas about how democracy should work.\footnote{E.g. Schulz, A., Müller, P., Schemer, C., Wirz, D. S., Wettstein, M., \& Wirth, W. (2018). Measuring populist attitudes on three dimensions. \textit{International Journal of Public Opinion Research}, 30(2), 316-326.} My project builds from findings in both of these veins: the former highlights the importance of looking at citizen attitudes toward concrete democratic decision-making processes, which in many instances may have little in common with abstract attitudes towards liberal democracy;\footnote{E.g. VanderMolen, K. (2017). Stealth Democracy Revisited: Reconsidering Preferences for Less Visible Government. \textit{Political Research Quarterly}, 70(3), 687-698.} while the latter underscores the value of looking at the individual-level, “demand-side” drivers of populism.\footnote{E.g. Van Hauwaert, S. M., \& Van Kessel, S. (2018). Beyond protest and discontent: A cross-national analysis of the effect of populist attitudes and issue positions on populist party support. \textit{European Journal of Political Research}, 57(1), 68-92.}

Combining insights from these literatures will allow this project to analyse the relative and interactive impact of economic and racial anxieties on democratic preferences, not only with regard to decision-making procedures, but also vis-à-vis broader populist trends. It will do so via careful methodological triangulation, involving a mix of quantitative and qualitative data collection measures spread across three working packages (WPs): a representative survey fielded in the United Kingdom, through which I will collect observational data and run several survey experiments (WP1); a series of lab experiments, designed to further unpack the mechanisms uncovered in the survey experiments (WP2); and focus groups and semi-structured interviews with a diverse set of subjects, which will allow me to examine the “everyday narratives” that tie economic vulnerability and racial resentment to concrete democratic processes (WP3).\footnote{See Stanley, L. (2016). Using focus groups in political science and international relations. \textit{Politics}, 36(3), 236–249.}

Taken as a whole, this project is intended to shed light on various issues of both academic and general interest. First, it aims to improve our understanding of popular preferences for how policymaking and policy-implementation processes should work, especially with regard to citizen involvement. Second, it sets out to disentangle the relative and interactive influence of economic and racial anxieties on these process preferences, focusing on concrete examples of decision-making procedures. Third, it seeks to connect these effects at the micro-level to issues of macro-level importance, examining their relationship to populist vote choice and dispositions. Given the societal and academic relevance of all three of these objectives, I believe that this project can make a timely contribution to ongoing political and academic debates.

\textbf{Alignment with the University’s Research Strategy}
\vspace{-.9em}

This proposal closely aligns with the University’s Global Research Theme on \textit{Developing Sustainable Societies} -- in particular the “Government and Policymaking” research group: by studying beliefs about how we should govern ourselves, the project will investigate the influence of various economic, social, and political challenges on process preferences. This alignment is further reinforced by the project’s focus on examining the desired roles of politicians, pressure groups, and stakeholders in policymaking and policy-implementation procedures. As a consequence, I believe that a Nottingham Research Fellowship would allow me to meaningfully contribute to the University’s pursuit of its research priorities.

\textbf{Planned Project Outputs}
\vspace{-.9em}

In the early stages of each WP, I will begin by soliciting feedback and disseminating initial results at conferences and departmental seminars. I will also publicize the status and findings of the project on my website and via submissions to newspapers, news magazines, and social science blogs with wide audiences. On the whole, the project will result in four scientific articles to be published in top-tier journals: WP1 will lead to two articles bringing together observational and experimental survey data, to be submitted to public opinion and political behaviour journals; while two subsequent articles, aimed at general political science journals, will combine survey data analyses with results from WP2 and WP3. Finally, each submission will have a corresponding report directed at NGOs and other experts, drawing out the study’s key policy implications.

\textbf{Career Aspirations}
\vspace{-.9em}

Aside from my goal of obtaining a permanent academic post, I aspire to produce research that appeals to an increasingly wide audience, both within academia and beyond it. By broaching research topics of broad theoretical and societal relevance, I hope to attain two objectives: first, to publish research in the top generalist journals within political science (e.g. the \textit{British Journal of Political Science}) as well as, eventually, with a top academic publishing press (e.g. Oxford University Press); and second, to have a tangible impact in politics by drawing the attention of journalists, NGOs, and politicians. Over the longer-run, I would also like to build upon these efforts by attracting research funding for multi-person projects from sources such as the Economic and Social Research Council, the Leverhulme Trust, or the British Academy.

\textbf{Career Implications of a Nottingham Research Fellowship}
\vspace{-.9em}

A Nottingham Research Fellowship would enhance my career prospects considerably, helping me to carry out innovative work that would extend my international research profile and expand the scope of my research programme. Moreover, the Fellowship’s generous research expense funding would deepen my experience with project management and research interventions by allowing me to collect a large quantity of data from a varied set of sources.

Finally, a position at the School of Politics and International Relations would, in and of itself, offer numerous career benefits. First, an affiliation at the School would give me the opportunity to consult and collaborate with leading, internationally-recognized scholars (such as Caitlin Milazzo and Anja Neundorf) who are interested in topics intricately related to my project. Second, consultations and training at the \textit{Methods and Data Institute} would help me to further develop my methodological abilities and meet academics using similar methods in related research areas. Third, holding a position at Nottingham would allow for collaborations and networking both within the University -- in particular via the \textit{Nottingham Interdisciplinary Centre for Economic and Political Research} and the \textit{Centre for British Politics} -- and outside of it -- most notably, at nearby Sheffield and Birmingham. A Nottingham Research Fellowship would thus offer real opportunities to expand my collaborative network and refine my research output by working alongside scholars who have published studies with my field’s top journals and book publishers.


\end{document}
