\documentclass[14pt]{beamer}
\setbeamertemplate{itemize item}{$\bullet$}
\setbeamertemplate{navigation symbols}{}
\setbeamertemplate{mini frames}{}
\setbeamertemplate{section in toc}[sections numbered]
\setbeamertemplate{itemize items}[square]
\setbeamercolor*{item}{fg=gray}
\usecolortheme{dove}
\hypersetup{pdfpagemode=FullScreen}

\usepackage[yyyymmdd]{datetime}

% Define some accent colors:
\definecolor{DarkFern}{HTML}{407428}
\definecolor{DarkCharcoal}{HTML}{4D4944}
\colorlet{Fern}{DarkFern!85!white}
\colorlet{Charcoal}{DarkCharcoal!85!white}
\colorlet{LightCharcoal}{Charcoal!50!white}
\colorlet{AlertColor}{orange!80!black}
\colorlet{DarkRed}{red!70!black}
\colorlet{DarkBlue}{blue!70!black}
\colorlet{DarkGreen}{green!70!black}
\definecolor{palecerulean}{rgb}{0.61, 0.77, 0.89}

% Use the colors:
\setbeamercolor{alerted text}{fg=palecerulean}
\setbeamercolor{title}{fg=blact}

% Create horizontal rule, header, and title page
\makeatletter
\def\vhrulefill#1{\leavevmode\leaders\hrule\@height#1\hfill \kern\z@}
\makeatother
\setbeamertemplate{frametitle}{\color{black}\bfseries\insertframetitle\par\vskip-6pt\color{palecerulean}\vhrulefill{1.75pt}}
\defbeamertemplate*{title page}{customized}[1][]
{
	{\begin{centering}
	\usebeamerfont{title}\bfseries\inserttitle\par\bigskip
	\usebeamerfont{subtitle}\insertsubtitle\par\vskip-3pt\color{palecerulean}\vhrulefill{1.75pt}\color{black}\\
	\bigskip
	\bigskip
	\usebeamerfont{author}\insertauthor\par\bigskip
	\usebeamerfont{institute}\insertinstitute\par
	\usebeamerfont{date}\insertdate\par
	\usebeamercolor[fg]{titlegraphic}\inserttitlegraphic
				\insertframetitle\par
	\smallskip\end{centering}}
}

\title{My Research and Potential Contributions to the Department}
\author{Anthony Kevins}
\institute{School of Governance\\
	Utrecht University}
\date{16-01-2019}


\begin{document}

\begin{frame}
\titlepage
\end{frame}

%Presentation: prepare a 5-7 minute presentation based on your work and contribution to the department.

%The Centre for Citizenship, Globalization and Governance (http://blog.soton.ac.uk/c2g2/about-us/): The Centre for Citizenship, Globalization and Governance (C2G2) focuses on the central political questions of today's world. We look at issues surrounding power, cooperation, security, inequality and democracy.

%Political Analysis Research Group (https://www.southampton.ac.uk/politics/research/groups/political-analysis.page?): The Political Analysis Research Group has a wide range of substantive interests and methodological expertise. Particular areas of research strength include: Political parties; Interest groups; Democratic innovations; Media and political communication; Public opinion and electoral behaviour; Comparative politics; Research methods.

%Citizenship, Justice and Democracy Research Theme (https://www.southampton.ac.uk/politics/research/themes.page): Problems of citizenship lie at the heart of many of the most important challenges facing the modern world. Whist the citizens of some countries are making unprecedented demands for democratic voice and the recognition of their civil and political rights, in others there is also mounting disillusionment with existing political processes and a questioning of the legitimacy of governments. In western democracies the current wave of austerity has provoked concerns that the social rights associated with the growth of welfare states are under growing attack, whilst many governments have responded to the financial crisis by demanding greater responsibility of their citizens. This cluster also examines inequality and justice across a number of dimensions, from local to global, and across dimensions (amongst others) of class, nationality, ethnicity and gender. It seeks to learn and integrate lessons from conceptual analysis and comparative politics, and to pay heed to new voices in debates about inequality and justice.

% A series of questions need to be addressed from two perspectives: citizens and policy makers. From the citizens’ perspective these include questions such as: what are the rights and responsibilities which citizenship entails? How is citizenship expressed? How do the discontented express themselves? How do civil society and protest contribute to social change? What is the role of new technologies for citizen participation? From the policy makers’ perspective questions include how can contemporary governments improve the legitimacy of their democracy? Is a deliberative system realistic? To bridge the divide between policy makers and citizens we will also engage with questions on the learning of citizenship and democracy and examine how and to what extent citizenship can be learned both inside and outside the formal education system.


\begin{frame}
\frametitle{Contributions}
\pause
Research contributions:
	\begin{itemize}
		\pause
		\item Societal Insiders and Outsiders
		\pause
		\item Welfare State Design and Expenditure
		\pause
		\item Social Policy Preferences
		\pause
		\item Connections to the \textit{Centre for Citizenship, Globalization and Governance}, the \textit{Citizenship, Justice and Democracy Research Theme}, and the \textit{Political Analysis Research Group} 
	\end{itemize}
\bigskip
\pause
Teaching contributions: 
\begin{itemize}
	\pause
	\item Alignment between past teaching and course offerings
	\pause
	\item Formal pedagogical training
\end{itemize}

\end{frame}

\begin{frame}
	\pause
	\frametitle{Societal Insiders and Outsiders}
	Consequences of social policy protection
	\begin{itemize}
		\pause
		\item E.g. Risk, social policy, and trust (Kevins, OnlineFirst)
	\end{itemize}
	\pause
	The impact of precarity
	\begin{itemize}
		\pause
		\item E.g. Precarity, the economy, and anti-immigrant sentiments (Kevins and Lightman, n.d.)
	\end{itemize}
	\pause
	Influence over the policy-making process
	\begin{itemize}
		\pause
		\item E.g. Low-income Americans and tax reforms (Kevins, n.d.)
	\end{itemize}
\end{frame}

\begin{frame}
	\frametitle{Welfare State Design and Expenditure}
	\pause
	Factors driving welfare reform
	\begin{itemize}
		\pause
		\item E.g. Changes in the size of the immigrant population (Soroka et al., 2016)
	\end{itemize}
	\pause
	Welfare state univeralism
	\begin{itemize}
		\pause
		\item E.g. Universalism and immigrant integration (Kevins and van Kersbergen, OnlineFirst)
	\end{itemize}
	\pause
	From welfare to workfare 
	\begin{itemize}
		\pause
		\item E.g. The effects of workfare on social solidarity (Kevins et al., n.d.)
	\end{itemize}
\end{frame}

\begin{frame}
	\pause
	\frametitle{Social Policy Preferences}
	The consequences of political discourse
	\begin{itemize}
		\pause
		\item E.g. Claims about benefit generosity (Jensen and Kevins, OnlineFirst)
	\end{itemize}
	\pause
	The relevance of identity markers 
	\begin{itemize}
		\pause
		\item E.g. Social class and attitudes toward the welfare state (Kevins et al., 2019)
	\end{itemize}
	\pause
	The nature of redistributive preferences
	\begin{itemize}
		\pause
		\item E.g. Assumptions about the motives driving pro- and anti-redistribution citizens (Kevins et al., Forthcoming)
	\end{itemize}
\end{frame}

\begin{frame}
	\frametitle{Teaching Contributions}
	\pause
	Alignment between teaching experience and course offerings
	\begin{itemize}
		\pause
		\item Democracy and Representation 
		\begin{itemize}
			\pause
			\item Political Systems
		\end{itemize}
		\pause
		\item Pragmatism and Politics
		\begin{itemize}
			\pause
			\item Governance and Policy
		\end{itemize}
		\pause
		\item Social Science Methods for Journalists
		\begin{itemize}
			\pause
			\item Research Skills in Politics \& International Relations
		\end{itemize}
	\end{itemize}
	\pause
	\bigskip
	Formal pedagogical training (e.g., 5-ECTS Pedagogical Programme for Assistant Professors)
	\begin{itemize}
		\pause
		\item Focus on constructive alignment, active learning, and an open dialogue with students
	\end{itemize}
\end{frame}

\end{document}
