\documentclass[letterpaper]{scrartcl}
\usepackage{tabularx}
\usepackage[none]{hyphenat}
\usepackage[utf8x]{inputenc}
\usepackage{lmodern,textcomp}
\usepackage{enumitem}
\usepackage{ltablex}
\usepackage[english]{babel}
\usepackage{blindtext}
\usepackage{microtype}
\usepackage[hidelinks]{hyperref}
\usepackage{hanging}
\usepackage[paperheight=279.4mm,paperwidth=210mm,bottom=10em]{geometry}
\usepackage{fancyhdr}
\setlength{\parskip}{1em}

\pagestyle{fancy}
\fancyhf{}
\fancyhead[LE,LO]{Teaching Philosophy}
\fancyhead[RE,RO]{Anthony Kevins}
\fancyfoot[CE,CO]{\leftmark}
\fancyfoot[CE,RO]{\thepage}
\setlength\parindent{0pt}
\setlist[itemize]{leftmargin=*}

\begin{document}

My teaching is shaped by the belief that a political science education has the potential to help form not only good political scientists, but also good citizens. In an age in which formulaic talking points often pass for quality political discourse, I believe that political science can help students move beyond facile arguments by improving their substantive political knowledge, scientific literacy, and critical reasoning skills. As a teacher, my task is to facilitate this growth by encouraging students to actively engage with both the course material and their peers. I do so by employing pedagogical best-practices tied to constructive alignment, active learning, and an open dialogue with students.

My course development centres around the notion that intended learning outcomes, active learning activities, and student assessment tasks should all be closely aligned (i.e. constructive alignment). The first step in this process is the careful defining of learning objectives. Outlining these goals helps to maintain consistency across the course materials and assessment methods, while also ensuring that students are familiar with the skills they are meant to develop. For example, in \textit{Political Institutions}, an undergraduate core course that I co-developed and co-taught, the compendium laid out the learning objectives for the course as a whole and for each of the individual weeks. By specifying in advance not only every lecture's theoretical and empirical objectives, but also the corresponding tutorial's theory-application objectives, we were able to ensure a clear thread throughout the semester. 

Having identified learning goals, I then develop individual lesson plans that allow me to build toward the intended learning outcomes in an engaging way. In a master's-level seminar course on \textit{Democracy and Representation}, for example, I opened each class with a brief discussion of a contemporary political issue connected to the lesson's topic (e.g., discussing British newspaper coverage of Jeremy Corbyn in a lesson looking at the media's role in shaping public opinion). Two goals guided this introductory framing: first, to tie the readings to current events and controversies, thereby highlighting the practical relevance of the course material; and second, to set the stage for theory-application later in the lesson, ensuring students have weekly opportunities to apply theories from the curriculum to real-world cases. 

Because this higher-level engagement requires a strong grasp of the assigned material, my lesson planning is also attuned to assuring student understanding. After connecting the week's topic to contemporary events or debates, I lay out a road map for the class by providing a set of questions they should be able to answer by the end of the day's session. I revisit key concepts over the course of the lesson through a combination of lecturing and active learning activities, having students, typically in pairs or small groups, engage with questions of understanding. These latter activities have taken a variety of formats, including online quizzes (followed by peer and then class-wide discussions), flipped classroom activities where students do the teaching, and reflections on the ``muddiest'' point of the lesson. These techniques allow me to ascertain how well students have grasped the course material while at the same time encouraging peer-to-peer explanation -- which, in turn, helps to facilitate active learning and overcome some of the difficulties inherent in the instructor-student knowledge gap.

Once students have a reasonable grasp of the material, I move on to more advanced learning objectives. While these goals vary based on the level and subject of a given course, this deeper exploration of a topic often involves several steps: applying an argument or theory to an event or case; analyzing the case from an alternative framework; and using the Socratic method to have students reflect upon any underlying assumptions. Furthermore, in-class discussions and debates are complemented by a variety of pre-class activities, including the preparation of small written assignments, group presentations, or methods exercises (with the aid of original screencasts for more technical tasks). These exercises help students to cement comprehension and refine the research, presentation, and writing skills that they will need after graduation.

My course design also incorporates various larger learning activities that punctuate the semester, with the aim of consolidating the skills and knowledge developed in individual lessons. In addition to the traditional graded assessments, my courses have included periodic stats labs, review sessions, and writing workshops. Some of these activities clearly necessitate more teacher intervention than others, but I always ensure that peer feedback plays a central role. In having students assist and assess one another, I seek to: (1) equip them with tools to more critically evaluate their own work; and (2) help them to develop skills that align with the course's learning objectives and the associated assessment tasks. At the same time, this approach also generates more frequent feedback for students and a more participatory classroom dynamic.

The final defining feature of my pedagogical approach is an open dialogue with students. Achieving this dialogue involves expressly making the case for the teaching activities I employ and then offering students the chance to provide informal, anonymous feedback after the first month of the semester. In my experience, this back-and-forth typically engenders a greater openness to new activities on their part. For example, some students have initially expressed doubt that peer feedback could be useful, based on the belief that only comments from their eventual grader are important; yet I have found that if I explain the motivations for this approach in advance, they quickly come to recognize and appreciate its benefits. The corollary of this, however, is that I must demonstrate an honest openness to adapting my teaching and learning activities when students are left unconvinced or have suggestions for improvements. 

Overall, my teaching practices are grounded in two beliefs: that a strong political science education can provide tools to make us better citizens; and that advances in pedagogical research can help us to more effectively reach our teaching goals. This approach has been at the core of each of my courses, whether at the undergraduate- or graduate-level; I have thus put it into practice in formats ranging from small seminars to large lectures, and teaching a variety of substantive and methodological topics. I look forward to a career in which I continually improve my ability to teach and engage students, building from their feedback while experimenting with new pedagogical techniques.


\end{document}
