\documentclass[letterpaper]{scrartcl}
\usepackage{tabularx}
\usepackage[none]{hyphenat}
\usepackage[utf8x]{inputenc}
\usepackage{lmodern,textcomp}
\usepackage{enumitem}
\usepackage{ltablex}
\usepackage[english]{babel}
\usepackage{blindtext}
\usepackage{microtype}
\usepackage[hidelinks]{hyperref}
\usepackage{hanging}
\usepackage[paperheight=279.4mm,paperwidth=210mm,bottom=10em]{geometry}
\usepackage{fancyhdr}
\setlength{\parskip}{1em}

\pagestyle{fancy}
\fancyhf{}
\fancyhead[LE,LO]{Anthony Kevins}
\fancyhead[RE,RO]{Research Interests}
\fancyfoot[CE,CO]{\leftmark}
\fancyfoot[CE,CO]{\thepage}
\setlength\parindent{0pt}
\setlist[itemize]{leftmargin=*}

\begin{document}

  Who deserves our collective support? This is a key question in contemporary democracies, and answers to it have varied dramatically across time, countries, and policy domains. On the one hand, social solidarity and generous social programmes seem to have existed in a virtuous circle of mutual reinforcement. Yet, that solidarity has always obscured some degree of ``us'' versus ``them'' distinctions -- often excluding immigrants, the ``non-deserving'' poor, or certain ethnic minorities -- and this tendency appears to have accentuated in recent years. Motivated by these tensions, my research explores the relationship between public policy, inequality, and the division between those who we deem worthy of solidarity and those who we exclude from it.

  My research programme centres around three questions that are fundamental to these discussions: What institutional factors shape relations between societal insiders and outsiders? What determinants drive the nature and timing of welfare state reform? And how can we best understand social policy preferences? In what follows, I lay out my research interests through a discussion of the work I have conducted on these questions.

  \textbf{(1) Societal Insiders and Outsiders}
  \vspace{-1em}

  First, my research aims to improve our understanding of how institutions affect the divide between groups that have historically benefited from a privileged access to state protection (i.e. insiders) and those that have typically been excluded from it (i.e. outsiders).

  My recent peer-reviewed book (Kevins, 2017) offers the clearest example of my interest in this topic. The monograph begins by comparing the introduction of coverage-extending policies in France and Italy, focusing on healthcare and income support for the unemployed. Combining elite interviews, archival research, and survey analysis, I argue that variation in benefit extension and standardization is more strongly influenced by institutional arrangements than by partisanship. While leftist parties have been the main drivers of reform, they do not consistently problematize gaps in coverage, since universalizing access to generous benefits typically entails fiscal and political trade-offs. I then assess the generalizability of these findings qualitatively (across six additional European cases) and quantitatively (across the OECD). Ultimately, the book highlights how the informal economy, welfare state institutions, and familialism shape the attention that coverage gaps receive from the public, trade unions, and political parties.

  My research on this theme extends to various article manuscripts as well. One paper leverages the contrasting trajectories of French and Italian minimum income schemes to draw out the factors shaping coverage extension (Kevins, 2015). Several other studies seek to unpack the attitudinal effects of the insider-outsider divide, arguing: (1) that labour market vulnerability and GDP levels -- but not unemployment rates -- interact to shape anti-immigrant attitudes (Kevins and Lightman, N.d.); and (2) that while labour market outsiders have lower levels of generalized trust than insiders, active labour market policy has the potential to minimize that divide (Kevins, 2018). Other ongoing work examines the causes and consequences of labour market exploitation (Kevins and Paraciani, N.d.) as well as the interactive effects of care work, professionalization, and inequality on job satisfaction (Lightman and Kevins, N.d.).

  Finally, my Marie Curie project extends my investigation of welfare state and labour market divisions to the divide between democratic insiders and outsiders. I do so via survey experiments in the US (already completed) and the Netherlands (to be conducted in early 2019) designed to study public opinion toward the influence of groups that are disproportionately impacted by a given policy decision (as, for example, with people of colour and policing policy). The project explores the circumstances under which citizens believe that policy-affected individuals should have more or less influence than others; to that end, I am currently investigating how Americans' reactions to consultation procedures are shaped by (a) racial resentment (Kevins and Robison, N.d.) and (b) the nature of individual policy preferences (Kevins, N.d.a.). Findings from this project will eventually also include a comparison of Dutch and American respondents as well as a co-authored piece (with Barbara Vis) exploring the impact of risk aversion on Dutch attitudes toward differentiated representation.

  \textbf{(2) Welfare State Design and Expenditure}
  \vspace{-1em}

  Unpacking the relationship between poverty, inequality, and solidarity in advanced democracies clearly also requires a nuanced understanding of the welfare state. A second strand of my research is thus focused on social programme design and expenditure.

  In addition to my monograph, which contains several chapters on the bi-directional interplay between welfare state design and public opinion (Kevins, 2017), I have numerous co-authored studies on this topic: one such article, based on corpus data from the Comparative Manifesto Project, uses welfare-specific items to push forward debates about how parties talk about the welfare state, equality, and social justice (Horn et al., 2017); while another uses the well-known case of 19th-century German social insurance legislation to explore the role of ``objective'' problem pressure on radical social policy innovation (Horn and Kevins, 2018). Other, ongoing work examines the extent to which parties attempt to tie the welfare state to clientelistic appeals to specific social groups (Horn et al., N.d.a.) and explores the relationship between party families and workfare reforms -- that is, policy changes that increase the emphasis on conditions and obligations for unemployed benefit seekers (Horn et al., N.d.b.).

  Increased migration has often been seen to pose a particular challenge for contemporary welfare states, and several of my studies thus focus on the connections between social policy, immigration, and solidarity. Research with Kees van Kersbergen, for example, unpacks the relationship between universalist policies, social cohesion, and migrant integration, using a comparison of Denmark and Canada to tease out a distinction between what we term \textit{community perks} and \textit{community scope} universalist traits (Kevins and van Kersbergen, Forthcoming). Another co-authored article argues that researchers looking for the effects of migration levels on aggregate social spending have obscured important insights: by examining domain-specific expenditure and changes rather than levels of migrant stock, we find that immigration primarily affects programmes that are subject to debates about moral hazard (Soroka et al., 2016).

  \textbf{(3) Public Opinion and Social Policy}
  \vspace{-1em}

  Finally, my research also seeks to develop our understanding of the public’s redistributive and social policy preferences, as well as the factors that influence them.

  Inequality has been central to much of my work on this topic, given both its clear potential ramifications and the recent increase in popular attention it has received. While scholars have debated the effects of overall inequality on redistributive preferences, I have led cross-national research that investigates how the structure of inequality matters in different ways for different income groups (Kevins et al., 2018b). The article's findings suggest that inequality systematically impacts preferences solely for the middle class – and even there it is only the distance between the middle and the upper-middle segments of the income distribution that has a robust effect. Other ongoing collaborative research uses an original dataset on workfare to examine the effects of these reforms on attitudes toward government support for the unemployed (Kevins et al., N.d.a.).

  These studies are complemented by an interest in the nature of social policy preferences. In an article with Stuart Soroka, we investigate long-term trends in Canadian attitudes toward redistribution, finding that the partisan sorting often discussed in the American context has distinct parallels in Canada (Kevins and Soroka, 2018). A paper based on survey experiment data collected with Carsten Jensen, in turn, examines the effects of assertions about benefit generosity on support for social assistance in the United Kingdom (Jensen and Kevins, 2018).

  Lastly, I also played a central role in designing an original survey on social policy preferences (fielded in nine European countries and the US as part of a project on Universalism and the Welfare State). I have led several studies based on this data: one paper uses novel survey items on respondents’ general relationship to the welfare state to argue that the importance of class markers for stances toward the welfare state has often been overstated in the literature (Kevins et al., 2018a); while another investigates cross-national patterns in the motives respondents ascribe to citizens who are pro- or anti-redistribution (Kevins et al., N.d.b.).

  \textbf{Conclusion}
  \vspace{-1em}

  Overall, my research aims to expand our understanding of how parties, public opinion, and institutions interact to shape social solidarity, especially in the face of increasing inequality, labour market segmentation, and economic austerity. To that end, my published and ongoing studies engage with debates in Comparative Politics about the effects of institutions on insiders and outsiders, the factors shaping welfare state design and expenditure, and the nature and determinants of social policy preferences. It is my hope, however, that the results of this research will have meaningful real-world implications as well -- most importantly, helping us to better understand why government policy varyingly reduces, reinforces, and generates inequality.

  \pagebreak

  \textbf{References:}
  \vspace{-1em}

  \begin{hangparas}{.25in}{1}

    Horn, Alexander, and Anthony Kevins. (2018) ``Problem Pressure and Social Policy Innovation: Lessons from 19th-Century Germany.'' \textit{Social Science History}, 42(3): 495-515.
    \vspace{-.5em}

    Horn, Alexander, Anthony Kevins, Carsten Jensen, and Kees van Kersbergen. (2017) ``Peeping at the Corpus: What is Really Going on Behind the Equality and Welfare Items of the Manifesto Project.'' \textit{Journal of European Social Policy}, 27(5): 403-416.
    \vspace{-.5em}

    Horn, Alexander, Anthony Kevins, Carsten Jensen, and Kees van Kersbergen. (N.d.a.) ``How Parties (Do Not) Appeal to Social Groups: Hollowing of Democracy or Hyper- Adaptive Targeting?''
    \vspace{-.5em}

    Horn, Alexander, Anthony Kevins, and Kees van Kersbergen. (N.d.b.) ``Partisanship and Workfare: Introducing and Applying a New Measure of Rights and Obligations in OECD Countries.''
    \vspace{-.5em}

    Jensen, Carsten, and Anthony Kevins. (2018) ``Numbers and Outrage in Welfare State Politics.'' OnlineFirst in \textit{Political Studies}.
    \vspace{-.5em}

    Kevins, Anthony. (2015) ``Political Actors, Public Opinion, and the Extension of Welfare Coverage.'' \textit{Journal of European Social Policy}, 25(3): 303-315.
    \vspace{-.5em}

    Kevins, Anthony. (2017) \textit{Expanding Welfare in an Age of Austerity: Increasing Protection in an Unprotected World}. Amsterdam, NL: Amsterdam University Press.
    \vspace{-.5em}

    Kevins, Anthony. (2018) ``Dualized Trust: Risk, Social Trust, and the Welfare State.'' OnlineFirst in \textit{Socio-Economic Review}.
    \vspace{-.5em}

    Kevins, Anthony. (N.d.) ``Who Should have a Say?: Preferences for Differentiated Representation.''
    \vspace{-.5em}

    Kevins, Anthony, Alexander Horn, and Kees van Kersbergen. (N.d.a.) ``Workfare and Public Opinion.''
    \vspace{-.5em}

    Kevins, Anthony, Alexander Horn, Carsten Jensen, and Kees van Kersbergen. (2018a) ``The Illusion of the Class in Welfare State Politics?'' OnlineFirst in the \textit{Journal of Social Policy}.
    \vspace{-.5em}

    Kevins, Anthony, Alexander Horn, Carsten Jensen, and Kees van Kersbergen. (2018b) ``Yardsticks of Inequality: Median Voter Preferences for Redistribution in Advanced Democracies.'' \textit{Journal of European Social Policy}, 28(4): 402-418.
    \vspace{-.5em}

    Kevins, Anthony, Alexander Horn, Carsten Jensen, and Kees van Kersbergen. (N.d.b.) ``Motive Attribution, Redistribution, and the Moral Politics of the Welfare State.''
    \vspace{-.5em}

    Kevins, Anthony, and Naomi Lightman. (N.d.) ``Immigrant Sentiment and Labour Market Vulnerability: Economic Perceptions of Immigration in Dualised Labour Markets.''
    \vspace{-.5em}

    Kevins, Anthony, and Rebecca Paraciani. (N.d.) ``Causes and Consequences of Labour Market Exploitation.''
    \vspace{-.5em}

    Kevins, Anthony, and Joshua Robison. (N.d.) ``Race and Differentiated Representation: The Case of Police Guidelines.'' \vspace{-.5em}

    Kevins, Anthony, and Stuart N. Soroka. (2018) ``Growing Apart? Partisan Sorting in Canada, 1992-2015.'' \textit{Canadian Journal of Political Science}, 51(1): 103-133.
    \vspace{-.5em}

    Kevins, Anthony, and Kees van Kersbergen. (Forthcoming) ``The Effects of Welfare State Universalism on Migrant Integration.'', \textit{Policy \& Politics}. \vspace{-.5em}

    Lightman, Naomi, and Anthony Kevins. (N.d.) ``Bonus or Burden?: Care Work and Job Satisfaction in Eighteen European Countries.''
    \vspace{-.5em}

    Soroka, Stuart N., Richard Johnston, Anthony Kevins, Keith Banting, and Will Kymlicka. (2016) ``Migration and Welfare State Spending.'' \textit{European Political Science Review}, 8(2): 174-193.

  \end{hangparas}

\end{document}
