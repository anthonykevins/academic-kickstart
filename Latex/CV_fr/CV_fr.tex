\documentclass[letterpaper,fontsize=10.5pt]{scrartcl}
\usepackage{tabularray}
\usepackage{hyphenat}
\usepackage[utf8x]{inputenc}
\usepackage{lmodern,textcomp}
\usepackage{enumitem}
\usepackage{ltablex}
\usepackage[french]{babel} 
\usepackage[paperheight=279.4mm,paperwidth=210mm,left=20mm,right=20mm,top=20mm,footskip=10mm,nohead]{geometry}
\usepackage{microtype}
\usepackage{blindtext}
\usepackage{titlesec}
\renewcommand{\thesection}{\Roman{section}}
\titleformat{\section}
{\normalfont\bfseries}{\thesection}{0em}{}
\renewcommand{\thesubsection}{\Roman{subsection}}
\titleformat{\subsection}
{\normalfont\mdseries\itshape}{\thesubsection}{0em}{}
\titlespacing*{\section}{0pt}{2ex plus 2ex minus 0ex}{1ex plus 0ex}
\titlespacing*{\subsection}{0pt}{2ex plus 2ex minus 0ex}{1ex plus 0ex}\setcounter{secnumdepth}{0}
\usepackage[sfdefault, light]{roboto}
\usepackage{newtxsf}
\usepackage{charter}
\usepackage{hyperref}
\hypersetup{
	colorlinks=true,
    urlcolor=[RGB]{56, 97, 246},
    pdfpagemode=FullScreen,
    unicode=false,
    pdftoolbar=true,
    pdfmenubar=true,
    pdffitwindow=false,
		pdfstartview={FitH},
    pdftitle={CV - Anthony Kevins},
    pdfauthor={Anthony Kevins},
    pdfsubject={Academic CV},
    bookmarksopen=true,
    bookmarksopenlevel=2,
    pdfstartview=Fit,
    pdfpagemode=UseOutlines,
	pdfnewwindow=true,
	pdfborder={0 0 0}
}
\usepackage{bookmark}

\setlength\parindent{0pt}
\setlength\parskip{0pt}
\setlist[itemize]{leftmargin=*}
\raggedbottom

\begin{document}

%Table formatting
\SetTblrInner{rowsep=0pt,colsep=5pt}
\SetTblrOuter{headsep=0pt,footsep=0pt,presep=0pt,postsep=0pt}
\DefTblrTemplate{contfoot-text}{none}{}
\SetTblrTemplate{contfoot-text}{none}
\DefTblrTemplate{conthead-text}{none}{}
\SetTblrTemplate{conthead-text}{none}

%HEADER

\huge Anthony Kevins\\
\begin{minipage}[t]{.87\linewidth}
	\large Professeur agrégé en science politique et d'études internationales\\
\end{minipage}%
\begin{minipage}[t]{.13\linewidth}
	\raggedleft
	\scriptsize \textbf{Janvier 2025}\\
\end{minipage}%
\normalsize

\hrulefill

\begin{minipage}[p]{.6\linewidth}
	\raggedright
	\medskip
	Morag Bell Building, Room 107\\
	Department of International Relations, Politics, and History\\
	Loughborough University, Epinal Way\\
	Loughborough, LE11 3TU, UK\\
	\medskip
\end{minipage}%
\begin{minipage}[p]{.4\linewidth}
	\raggedleft
	\medskip
	Téléphone : \href{tel:+4401509228367}{+44 (0)1509 228367}\\
	Courriel : \href{mailto:a.kevins@lboro.ac.uk}{a.kevins@lboro.ac.uk}\\
	Site web : \href{https://anthonykevins.github.io/fr/}{anthonykevins.github.io/}\\
	ORCID: \href{https://orcid.org/0000-0003-3172-6632}{0000-0003-3172-6632}\\
	\medskip
\end{minipage}%

\hrulefill

%ACADEMIC POSITIONS
\section{Postes académiques}
\vspace{-2em}
\begin{longtblr}[entry=none,label=none]{@{} l X @{}}
	2019-     & Université de Loughborough, Département des relations internationales, des sciences politiques et d'histoire \\ [-.5ex]
		& \textbullet\ Professeur agrégé en science politique et d'études internationales, 2024-\\[-.5ex]
		& \textbullet\ Professeur adjoint en science politique et d'études internationales, 2019-2023\\
	2017-2019 & Université d'Utrecht, École de gouvernance                                             \\[-.5ex]
	& \textbullet\ Chercheur Marie Curie \\
	2014-2017 & Université d'Aarhus, Département de science politique                               \\ [-.5ex]
		& \textbullet\ Professeur adjoint (CDD) \\
\end{longtblr}

%EDUCATION
\section{Formation universitaire}
\vspace{-2em}
\begin{longtblr}[entry=none,label=none]{@{} l X @{}}
	2008-2014 & Doctorat en science politique, Université McGill                                                                  \\
	2006-2007 & Maîtrise en politique comparée, London School of Economics and Political Science                                 \\ 
	2002-2006 & Baccalauréat (hons.) bidisciplinaire en science politique, droit et société, Université York     \\              
\end{longtblr}

%RESEARCH INTERESTS
\section{Domaines d’expertise}
\vspace{-2em}
\begin{longtblr}[entry=none,label=none]{@{}X}
	Opinion publique, Représentation démocratique, Inégalités, Politiques sociales, Précarité sur le marché du travails. \\
\end{longtblr}

%GRANTS AND FELLOWSHIPS
\section{Prix et bourses}
\vspace{-2em}
\begin{longtblr}[entry=none,label=none]{@{} l X @{} }
	2024-2025 & UK Young Academy Project Funding (collaborateur de projet), £13 850 \\
	2023-2025 & Small Research Grant, The British Academy/Leverhulme Trust, £9930 (231164) \\ 
	2023 & SSH Research and Innovation Funding, École des sciences humaines et sociales, Université de Loughborough, £1997 \\
	2021-2022 & Early Career Researcher Seed Corn Funding, École des sciences humaines et sociales, Université de Loughborough, 2000 £ \\ 
	2017-2019 & Bourse individuelle Marie Curie, Horizon 2020, programme cadre de l'Union européenne, 165 598 €                                                                  \\
	2017      & Departmental Research Grant, Département de science politique, Université d'Aarhus (avec Carsten Jensen), 45 000 DKK                                               \\
	2013-2014 & Bourse d'études de l’Institut des politiques sociales et de la santé de McGill, 5000 \$                   \\
	2010-2013 & Bourse Guy Drummond, Faculté des arts, Université McGill, 37 500 \$                                                                                            \\
	2010      & Bourse pour participer au Consortium on Qualitative Research Methods, Département de science politique, Université McGill                                             \\
	2008-2010 & Bourse Nathan Steinberg en science politique, Faculté des arts, Université McGill, 30 000 \$                                                                   \\
	2008-2009 & Bourse du doyen au troisième cycle, Faculté des arts,Université McGill, 2000 \$                                                                              \\
\end{longtblr}

%PUBLICATIONS
\section{Publications}
\subsection{Livres (avec comité de lecture)}
\vspace{-2em}
\begin{longtblr}[entry=none,label=none]{@{} l X @{}}
	2017 & Anthony Kevins. \textit{Expanding Welfare in an Age of Austerity: Increasing Protection in an Unprotected World}, Amsterdam University Press (Série : Changing Welfare States). \href{https://repository.lboro.ac.uk/articles/book/Expanding_welfare_in_an_age_of_austerity_Increasing_protection_in_an_unprotected_world/9994709}{Dépôt institutionnel} \\ 
\end{longtblr}

\subsection{Articles (dans des revues avec comité de lecture)}
\vspace{-2em}
\begin{longtblr}[entry=none,label=none]{@{} l X @{}}
	n.d. & Anthony Kevins et Joshua Robison. « Do the Origins of Climate Assemblies Shape Public Reactions? Examining the Impact of Partisanship », \textit{European Journal of Political Research} (OnlineFirst). \href{https://doi.org/10.1111/1475-6765.12743}{doi.org/10.1111/1475-6765.12743}.

	\\

	2025 & Anthony Kevins et Naomi Lightman. « Satisfaction with Social Care in the UK: Assessing the Interactive Effects of Age and Ideology », \textit{International Journal of Social Welfare}, 34(1): e12710. \href{https://doi.org/10.1111/ijsw.12710}{doi.org/10.1111/ijsw.12710}

	\\

	2024 & Alexander Horn, Anthony Kevins et Kees van Kersbergen. « Workfare and Attitudes toward the Unemployed: New Evidence on Policy Feedback from 1990 to 2018 », \textit{Comparative Political Studies}, 57(5), 818-850. \href{https://doi.org/10.1177/00104140231178743}{doi.org/10.1177/00104140231178743}                    
	\\
	
	2023 & Alexander Horn, Anthony Kevins et Kees van Kersbergen. ``The Paternalist Politics of Punitive and Enabling Workfare: Evidence from a New Dataset on Workfare Reforms in 16 Countries, 1980-2015'', \textit{Socio-Economic Review}, 21(4): 2137–2166. \href{https://doi.org/10.1093/ser/mwac060}{doi.org/10.1093/ser/mwac060}                                                          \\ [-.5ex]
	& $\circ$ Récipiendaire du \href{https://sase.org/publications/socio-economic-review/ser-best-paper-awards/}{SER Best Article Prize} pour le meilleur article publié en 2023 dans \textit{Socio-Economic Review}.                                                                                                                                                                                                                                                    \\
	
	2023 & Anthony Kevins et Barbara Vis. « Do Public Consultations Reduce Blame Attribution? The Impact of Consultation Characteristics, Gender, and Gender Attitudes », \textit{Political Behavior}, 45: 1121–1142. \href{https://doi.org/10.1007/s11109-021-09751-5}{doi.org/10.1007/s11109-021-09751-5}                                             \\
	
	2023 & Anthony Kevins et Seonghui Lee. « Projection in the Face of Centrism: Voter Inferences about Candidates’ Party Affiliation in Low-information Contexts », \textit{Political Psychology}, 44(2): 319-336. \href{https://doi.org/10.1111/pops.12851}{doi.org/10.1111/pops.12851}                                                             \\

	2022 & Anthony Kevins. « Distributing Democratic Influence: External Efficacy and the Preferred Influence of Policy Winners and Losers », \textit{International Journal of Public Opinion Research}, 34(4): 1-11. \href{https://doi.org/10.1093/ijpor/edac035}{doi.org/10.1093/ijpor/edac035}                                                                                                                                              \\

	2022          & Anthony Kevins. « Input from Whom? Public Reactions to Different Consultation Measures », \textit{Political Studies}, 70(2): 281-303. \href{https://doi.org/10.1177/0032321720956327}{doi.org/10.1177/0032321720956327}                                                                                                                     \\
	2022          & Anthony Kevins et Naomi Lightman. ``How Should the Government Treat Asylum Seekers? The Role of Labour Market Vulnerability and Ethnic Diversity in Europe'', \textit{Social Science Research}, 104. \href{https://doi.org/10.1016/j.ssresearch.2021.102666}{doi.org/10.1016/j.ssresearch.2021.102666}                                        \\
	2021 & Alexander Horn, Anthony Kevins, Carsten Jensen et Kees van Kersbergen. « Political Parties and Social Groups: New Perspectives and Data on Group and Policy Appeals », \textit{Party Politics}, 27(5): 983–995. \href{https://doi.org/10.1177/1354068820907998}{doi.org/10.1177/1354068820907998}                                             \\
	
	2021          & Naomi Lightman et Anthony Kevins. « `Women's Work': Welfare State Spending and the Gendered and Classed Dynamics of Unpaid Care ».  \textit{Gender \& Society}, 35(5): 778-805. \href{https://doi.org/10.1177/08912432211038695}{doi.org/10.1177/08912432211038695}                                                                         \\
	2021          & Anthony Kevins. « Race, Class, or Both? Responses to Candidate Characteristics in Canada, the UK, and the US », \textit{Politics, Groups, and Identities}, 9(4): 699-720. \href{https://doi.org/10.1080/21565503.2019.1636833}{doi.org/10.1080/21565503.2019.1636833}                                                                       \\
	& $\circ$ Couverture médiatique dans le \href{https://vancouversun.com/opinion/columnists/women-people-colour-get-fewer-votes-canada-studies}{Vancouver Sun}.                                                                                                    \\

	2021          & Anthony Kevins et Joshua Robison. ``Who Should Get a Say? Race, Law Enforcement Guidelines, and Systems of Representation'', \textit{Political Psychology}, 42(1): 71-91. \href{https://doi.org/10.1111/pops.12688}{doi.org/10.1111/pops.12688}                                                                                               \\
	2020          & Anthony Kevins et Naomi Lightman. « Immigrant Sentiment and Labour Market Vulnerability: Economic Perceptions of Immigration in Dualized Labour Markets », \textit{Comparative European Politics}, 18(3): 460–484. \href{https://doi.org/10.1057/s41295-019-00194-1}{doi.org/10.1057/s41295-019-00194-1}                                  \\
	2020          & Anthony Kevins, Alexander Horn, Carsten Jensen et Kees van Kersbergen. « Motive Attribution and the Moral Politics of the Welfare State », \textit{Journal of Social Policy}, 49(1): 145-165. \href{https://doi.org/10.1017/S0047279419000175}{doi.org/10.1017/S0047279419000175}                                                           \\
	2019          & Naomi Lightman et Anthony Kevins. « Bonus or Burden? Care Work, Inequality, and Job Satisfaction in Eighteen European Countries », \textit{European Sociological Review}, 35(6): 825–844. \href{https://academic.oup.com/esr/article/35/6/825/5521386?guestAccessKey=5a546076-ebad-417e-a168-d998e6b56a96}{doi.org/10.1093/esr/jcz032}    \\
	2019          & Anthony Kevins. « Dualized Trust: Risk, social trust, and the Welfare State », \textit{Socio-Economic Review}, 17(4): 875–897. \href{https://doi.org/10.1093/ser/mwx064}{doi.org/10.1093/ser/mwx064}                                                                                                                                      \\	
	2019          & Carsten Jensen et Anthony Kevins. « Numbers and Attitudes Towards Welfare State Generosity », \textit{Political Studies}, 67(2): 496–516. \href{https://doi.org/10.1177/0032321718780516}{doi.org/10.1177/0032321718780516}                                                                                                               \\ [-.5ex]
	              & $\circ$ Récipiendaire du \href{https://journals.sagepub.com/page/psx/collections/virtual-special-issues/harrison-prize-winners}{Harrison Prize} pour le meilleur article publié en 2019 dans \textit{Political Studies}.                                                                                                                                                                                                                                                    \\
	2019          & Anthony Kevins et Kees van Kersbergen. « The Effects of Welfare State Universalism on Migrant Integration », \textit{Policy \& Politics}, 47(1): 115-132. \href{https://doi.org/10.1332/030557318X15407315707251}{doi.org/10.1332/030557318X15407315707251}                                                                                 \\
	2019          & Anthony Kevins, Alexander Horn, Carsten Jensen et Kees van Kersbergen. « The Illusion of Class in Welfare State Politics? », \textit{Journal of Social Policy}, 48(1): 21-41. \href{https://dx.doi.org/10.1017/S0047279418000247}{dx.doi.org/10.1017/S0047279418000247}                                                                     \\ 
	2018          & Anthony Kevins, Alexander Horn, Carsten Jensen et Kees van Kersbergen. « Yardsticks of Inequality: Preferences for Redistribution in Advanced Democracies », \textit{Journal of European Social Policy}, 28(4): 402-418. \href{https://doi.org/10.1177/0958928717753579}{doi.org/10.1177/0958928717753579}                                  \\ 
	2018          & Alexander Horn et Anthony Kevins. « Problem Pressure and Social Policy Innovation: Lessons from 19th-Century Germany », \textit{Social Science History}, 42(3): 495-515. \href{https://doi.org/10.1017/ssh.2018.13}{doi.org/10.1017/ssh.2018.13}                                                                                            \\ 
	2018          & Anthony Kevins et Stuart Soroka. « Growing Apart? Partisan Sorting in Canada, 1992-2015 »,  \textit{Canadian Journal of Political Science}, 51(1): 103-133. \href{https://doi.org/10.1017/S0008423917000713}{doi.org/10.1017/S0008423917000713}                                                                                             \\ [-.5ex]
	              & $\circ$ Couverture médiatique dans \href{https://www.theglobeandmail.com/opinion/big-tent-politics-is-now-all-but-dead/article24944734/}{The Globe and Mail} et \href{https://www.macleans.ca/politics/this-is-whats-wrong-with-canadas-right/}{Maclean's}.                                                                                          \\
	2017          & Alexander Horn, Anthony Kevins, Carsten Jensen et Kees van Kersbergen. « Peeping at the Corpus – What is Really Going on behind the Equality and Welfare Items of the Manifesto Project? », \textit{Journal of European Social Policy}, 27(5): 403-416. \href{https://doi.org/10.1177/0958928716688263}{doi.org/10.1177/0958928716688263} \\ 
	2016          & Stuart Soroka, Richard Johnston, Anthony Kevins, Keith Banting et Will Kymlicka. « Migration and Welfare State Spending », \textit{European Political Science Review}, 8(2): 173-194. \href{https://doi.org/10.1017/S1755773915000041}{doi.org/10.1017/S1755773915000041}                                                                   \\ 
	2015          & Anthony Kevins. « Political Actors and the Extension of Welfare Coverage », \textit{Journal of European Social Policy}, 25(3): 303-315. \href{https://doi.org/10.1177/0958928715588705}{doi.org/10.1177/0958928715588705}                                                                                                                   \\
\end{longtblr}

\subsection{Chapitres de Livre}
\vspace{-2em}
\begin{longtblr}[entry=none,label=none]{@{} l X @{}}
	2024 & Anthony Kevins et Barbara Vis. « Blame, Public Consultations, and the Impact of Gender », \textit{The Politics and Governance and Blame}, edité par Matthew Flinders, Gergana Dimova, Markus Hinterleitner, R. A. W. Rhodes et R. Kent Weaver, Oxford University Press: 732-766. \href{https://doi.org/10.1093/oso/9780198896388.003.0029}{doi.org/10.1093/oso/9780198896388.003.0029} \\
	2023 & Alexander Horn et Anthony Kevins, « Ever the Committed Egalitarians – or the End of Scandinavian Exceptionalism? Comparing Equality and Welfare State Preferences among Voters and Parties », \textit{No Normal Science! Festschrift for Kees van Kersbergen}, edité par Christoffer Green-Pedersen, Carsten Jensen, and Barbara Vis, Politica: 160-172. \href{https://repository.lboro.ac.uk/articles/chapter/Ever_the_committed_egalitarians_or_the_end_of_Scandinavian_exceptionalism_Comparing_equality_and_welfare_state_preferences_among_voters_and_parties/24220813}{Dépôt institutionnel}\\
	2022 & Anthony Kevins. « The Impact of Labour Market Vulnerability: Explaining Attitudes toward Immigration in Europe », \textit{Comparative Public Opinion}, edité par Cameron D. Anderson et Mathieu Turgeon, Routledge: 259-283. \href{https://doi.org/10.4324/9781003121992-17}{doi.org/10.4324/9781003121992-17}                             \\
	2022 & Anthony Kevins. « When Does Immigration Shape Support for a Universal Basic Income? The Role of Education and Employment Status », \textit{The Handbook on Migration and Welfare}, edité par Markus M. L. Crepaz, Edward Elgar Publishing: 137-155. \href{https://doi.org/10.4337/9781839104572.00014}{doi.org/10.4337/9781839104572.00014} 
\end{longtblr}

%ONGOING PROJECTS
% \section{Projets en cours}
% \begin{itemize}[itemsep=0em, topsep=0em, partopsep=0em]
% 	\item « The Consequences of Care: Psychological Wellbeing of Unpaid and Paid Carers and the Role of Social Expenditure », avec Naomi Lightman.
% 	\item « How and When Does Workfare Shape Policy Attitudes? New Evidence on Cross-National Trends in Workfare and Policy Preferences from 1990 to 2018 », avec Alexander Horn et Kees van Kersbergen.
% 	\end{itemize}

%CONFERENCE PAPERS (LAST FIVE YEARS)
\section{Présentations (5 dernières années)}
\vspace{-2em}
\begin{longtblr}[entry=none,label=none]{@{} l X @{}}
	2024 & « How We Think About the Political Stances of Others: Evidence on Projection from Canada, Germany, and the UK »,avec Seonghui Lee. Elections, Public Opinion and Parties Conférence, Manchester, Royaume-Uni. \\
	2024 & « Optimal Representation: the Importance of Identity in European Parliament Elections », avec William Daniel. Conférence annuelle de l'University Association for Contemporary European Studies, Trente, Italie. \\
	2023 & « How We Think About the Political Stances of Others: Evidence on Projection from Canada, Germany, and the UK », avec Seonghui Lee. Conférence annuelle de l'European Political Science Association, Glasgow, Royaume-Uni. \\ 
	2023 & « Investigating the Democratic Legitimacy of Citizens Climate Assemblies », avec Joshua Robison. Conférence annuelle de l'European Political Science Association, Glasgow, Royaume-Uni. \\ 
	2023 & « The Consequences of Care: Psychological Well-being of Unpaid and Paid Carers and the Role of Social Expenditure », avec Naomi Lightman. Global Carework Summit, San José, Costa Rica. \\
	2023 & « Understanding the Legitimacy of Deliberative Mini-Publics », avec Joshua Robison. Conférence annuelle de la Midwestern Political Science Association, Chicago, IL. \\
	2023 & « How We Think About the Political Stances of Others: Evidence on Projection from Canada, Germany, and the UK », avec Seonghui Lee. Conférence annuelle de la Political Studies Association, Liverpool, Royaume-Uni. \\
	2022 & « Distributing Democratic Influence: Political Efficacy and the Preferred Influence of Policy Winners and Losers ». Conférence annuelle de la Political Studies Association (en ligne).\\
	2022 & « Projection in the Face of Centrism: Voter Inferences about Candidates’ Party Affiliation in Low-information Contexts », avec Seonghui Lee. Conférence annuelle de la Political Studies Association (en ligne).\\
	2021 & « Do Public Consultations Reduce Blame Attribution? 
	Studying the Impact of Consultation Characteristics, Gender, and Gender Attitudes », avec Barbara Vis. Conférence annuelle de l'American Politics Group, Political Studies Association (en ligne).\\		
	2020 & « How Workfare Shapes Social Solidarity: Cross-National Trends in Workfare and Public Opinion Since the 1980s », avec Alexander Horn et Kees van Kersbergen. Conférence générale de l'ECPR (en ligne).\\
	%2019 & « Input from Whom? Public Opinion, Tax Reform, and Representation ». Conférence annuelle de l'EPSA, Belfast, Irlande du Nord.\\
	%2019 & « Immigrant Sentiment and Labour Market Vulnerability: Economic Perceptions of Immigration in Dualized Labour Markets », avec Naomi Lightman. Conférence annuelle de l'EPSA, Belfast, Irlande du Nord.\\
	%2019 & « Public Consultations: Do Weakened Clarity of Responsibility and Increased Perceived Responsiveness Reduce Blame Attribution? », avec Barbara Vis. Conférence annuelle de l'Association néerlandaise-flamande de science politique, Anvers, Belgique.\\		
	%2019 & « Bonus or Burden? Care Work, Inequality, and Job Satisfaction in Eighteen European Countries », avec Naomi Lightman. Conférence annuelle de l'Association canadienne de sociologie, Vancouver, Canada.\\ 
	%2019 & « Public Consultations: Do Weakened Clarity of Responsibility and Increased Perceived Responsiveness Reduce Blame Attribution? », avec Barbara Vis. « UvA Experimental Design Group/Hot Politics Lab », Amsterdam, Pays-Bas.\\		
	%2019 & « Exploring Voter Reactions to Public Consultations: Race, Issue Publics, and Blame Attribution ». « Sociology Colloquium », Tilbourg, Pays-Bas.\\
	%2019 & « The Politics of Conditional Solidarity: Developments and Political Determinants of the Workfare Balance in 16 Countries, 1980-2015 », avec Alexander Horn et Kees van Kersbergen. Danish Centre for Welfare Studies « Early Career Workshop », Odense, Danemark.\\
	%2019 & « The Structure of Inequality and Support for Redistribution », avec Henning Finseraas. Atelier « Inequalities and Preference for Redistribution », Paris, France. \\
	%2019 & « The Politics of Conditional Solidarity: Developments and Political Determinants of the Workfare Balance in 16 Countries, 1980-2015 », avec Alexander Horn et Kees van Kersbergen. Conférence en politique comparée de l'Association allemande de science politique, Munich, Allemagne.\\
	% 2018 & « Immigrant Sentiment and Labour Market Vulnerability: Economic Perceptions of Immigration in Dualized Labour Markets », avec Naomi Lightman. Conférence « Challenges to European Integration: Welfare States and Free Movement in the EU », Leyde, Pays-Bas.\\
	% 2018 & « Who Should Have a Say?: Preferences for Differentiated Representation ». Conférence annuelle de l'Association américaine de science politique, Boston, MA.\\ 
	% 2018 & « Bonus or Burden? Care Work and Job Satisfaction in Eighteen European Countries », avec Naomi Lightman. Conférence annuelle de l'Association américaine de sociologie, Philadelphie, PA.\\ 
	% 2018 & « Who Should Have a Say?: Preferences for Differentiated Representation ». Conférence annuelle de l'Association néerlandaise-flamande de science politique, Leyde, Pays-Bas.\\ 
	% 2017 & « Who Should Have a Say?: Preferences for Unequal Representation ». Conférence invitée, « Democratic Representation », Aarhus, Danemark.\\ 
	% 2017 & « The Illusion of the Middle Class in Welfare State Politics », avec Alexander Horn, Carsten Jensen et Kees van Kersbergen. Conférence annuelle de la CES, Glasgow, Écosse.\\ 
	% 2017 & « Perceived Class Position and the Welfare State », avec Alexander Horn, Carsten Jensen et Kees van Kersbergen. Atelier « The Middle class and the Welfare State », Aarhus, Danemark.\\ 
	% 2016 & « Selfish, Good-hearted, or Nasty? Motive Attribution and the Moral Economy of the Welfare State », avec Alexander Horn, Carsten Jensen et Kees van Kersbergen. Conférence annuelle de l'Association américaine de science politique, Philadelphie, PA.\\ 
	% 2016 & « How Parties (Do Not) Appeal to Social Groups », avec Alexander Horn. Conférence annuelle de l'Association danoise de science politique, Vejle, Danemark. \\ 
	% 2015 & « Inequality, Parties, and Public Opinion », avec Alexander Horn. Conférence du Réseau d’études sur les politiques d'inégalité, Odense, Danemark.  \\ 
	% 2015 & « The Special Path? Asymmetric Federalism, Problem Pressure, and the Development of the German Welfare State », avec Alexander Horn. Conférence annuelle de l'Association danoise de science politique, Kolding, Danemark.\\ 
	% 2015 & « Yardsticks of Inequality », avec Carsten Jensen, Kees van Kersbergen et Alexander Horn. Conférence annuelle de la CES, Paris, France.\\ 
	% 2015 & « Peeping at the Corpus: What is Really Going on Behind the Equality and Welfare Items of the Manifesto Project? », avec Alexander Horn, Kees van Kersbergen et Carsten Jensen. « Manifesto Project User Conference », Berlin, Allemagne.\\ 
	% 2015 & « Yardsticks of Inequality », avec Carsten Jensen, Kees van Kersbergen et Alexander Horn. Conférence annuelle de la SASE, Londres, Angleterre.\\ 
	% 2015 & « The Welfare State, Statecraft, and Migration: Social Policy and Integration in Denmark and Canada », avec Kees van Kersbergen et Carsten Jensen. Conférence annuelle de l'ACSP, Ottawa, Canada. \\ 
	% 2014 & « The Strivers and the Skivers: Public Opinion, Political Discourse, and Changes to Benefit Access ». Conférence annuelle de l'ACSP, St. Catharines, Canada.\\ 
	% 2014 & « Redistributive Preferences and Partisan Polarization: Canada in Comparative Perspective », avec Stuart Soroka. Conférence « Canadianizing the United States? Public Opinion across the 49th Parallel », Berkeley, CA.\\ 
	% 2014 & « The Strivers and the Skivers: Public Opinion, Political Discourse, and Changes to Benefit Access ». Conférence annuelle de la CES, Washington, DC.\\ 
	% 2013 & « Extending and Standardising Care: Healthcare Reform in France and Italy ». Conférence annuelle de la CES, Amsterdam, Pays-Bas.\\ 
	% 2013 & « Migration and Welfare State Spending », avec Stuart Soroka, Richard Johnston, Keith Banting et Will Kymlicka. Conférence annuelle de la CES, Amsterdam, Pays-Bas.\\ 
	% 2013 & « Migration and Welfare State Spending », avec Stuart Soroka, Richard Johnston, Keith Banting et Will Kymlicka. Conférence annuelle de l'ACSP, Victoria, Canada.\\ 
	% 2013 & « Political Actors and the Extension of Welfare Coverage: Benefits for the Unemployed in France and Italy ». Conférence annuelle de l'ACSP, Victoria, Canada.\\
\end{longtblr}

%TEACHING
\section{Enseignement}
\phantomsection
\addcontentsline{toc}{subsection}{1\textsuperscript{er} cycle}\textit{1\textsuperscript{er} cycle}: British Politics and Government (Université de Loughborough); Comparative European Politics (Université de Loughborough); Comparative Political Economy (Université de Loughborough); Dissertation (Université de Loughborough); Introduction to Philosophy (Université de Loughborough); Political Institutions: Western Countries, the European Union and International Institutions (Aarhus University); Politics: Contemporary Europe (Université McGill); Research Design (Université de Loughborough); Smart Scholarship (Université de Loughborough).\\
\hfill \break
\phantomsection
\addcontentsline{toc}{subsection}{2\textsuperscript{e} cycle}\textit{2\textsuperscript{e} cycle}: Democracy and Representation (Aarhus University); Pragmatism and Politics (Aarhus University); Social Science Methods for Journalists (Aarhus University).

%PROFESSIONAL TRAINING
\section{Formation professionnelle}
\phantomsection
\addcontentsline{toc}{subsection}{Formation générale}\textit{Formation générale} : « New Lecturers Programme », Université de Loughborough (2019-2022).\\
\hfill \break
\phantomsection
\addcontentsline{toc}{subsection}{Formation pédagogique}\textit{Formation pédagogique} : « Supervisor Development Programme », Université de Loughborough ; « Recognition of Experienced Practitioners Programme », Université de Loughborough (2020-2021); Programme de formation d'enseignants pour les professeurs adjoints, Université d'Aarhus (2016-2017) ; Cours pédagogique sur les défis de la classe multiculturelle danoise, Université d'Aarhus (2016).\\
\hfill \break
\phantomsection
\addcontentsline{toc}{subsection}{Formation méthodologique}\textit{Formation méthodologique} : Ateliers de formation à l'Essex Summer School in Social Science Data Analysis (2015), au Centre pour l’étude de la citoyenneté démocratique (2013-2014) et	à l'Institute for Qualitative and Multi-Method Research (2010).\\
\hfill \break
\phantomsection
\addcontentsline{toc}{subsection}{Formation linguistique}\textit{Formation linguistique} : Cours de français – niveau autonome (C1), Alliance française (2018-2019) ; Certificat de compétence en français écrit – Communication en milieu de travail, Université McGill (2017) ; Cours de danois, Lærdansk (2014-2016) ; Explore – programme d'immersion (La Pocatière, Québec), Campus Saint-Jean, Université d'Alberta (2010).\\
\hfill \break
\phantomsection
\addcontentsline{toc}{subsection}{Logiciels}\textit{Logiciels} : Stata ; R ; Git ; Qualtrics ; \LaTeX\ ; Markdown.

%PROFESSIONAL SERVICE
\section{Service à la communauté}
\phantomsection
\addcontentsline{toc}{subsection}{Engagement communautaire}\textit{Engagement communautaire} : \href{https://valonline.org.uk/future-focus-2024/}{Voluntary Action LeicesterShire} (2024); \href{https://issuu.com/kemps/docs/biz_network_april_2024/74}{Business Network} (2024), \href{https://senedd.wales/seneddclimate}{The Welsh Parliament's Climate Change, Environment and Infrastructure Committee} (2022), \href{https://tinyurl.com/4t6k26nh/}{The Political Behavior Blog} (2021), \href{https://stukroodvlees.nl/meer-inspraak-minder-schuld/}{Stuk Rood Vlees} (2021); \href{http://www.wipsociology.org/2021/09/16/womens-work-and-the-welfare-state-new-analysis-quantifies-how-gender-class-and-social-policy-shape-unpaid-care-work/}{Work In Progress}/\href{https://gendersociety.wordpress.com/2021/09/03/womens-work-and-the-welfare-state-new-analysis-quantifies-how-gender-class-and-social-policy-shape-unpaid-care-work/}{The Gender \& Society Blog} (2021),  \href{https://www.lboro.ac.uk/media-centre/press-releases/2020/october/life-beyond-covid-19-loughborough-experts/}{The Parliamentary Office of Science and Technology's (POST) Knowledge Exchange Unit} (2020),  \href{https://socialpolicyblog.com/2019/05/08/explaining-other-peoples-stances-on-inequality/}{The Social Policy Blog} (2019), \href{https://cordis.europa.eu/project/rcn/209009/brief/fr}{Le service d’information sur la recherche et le développement communautaires (CORDIS)} (2019), \href{https://www.lemonde.fr/idees/article/2019/03/22/nous-demandons-des-programmes-sociaux-moins-genereux-lorsque-nos-revenus-diminuent_5439877_3232.html}{Le Monde} (2019), \href{https://archive.discoversociety.org/2019/02/06/policy-and-politics-one-of-us-how-welfare-states-help-shape-immigrant-integration/}{Discover Society} (2019), \href{https://policyandpoliticsblog.com/2019/02/20/one-of-us-how-welfare-states-help-shape-immigrant-integration/}{Policy \& Politics Journal Blog} (2019) et \href{http://blogs.lse.ac.uk/politicsandpolicy/how-claims-about-welfare-benefit-levels-affect-public-opinion/}{LSE British Politics \& Policy Blog} (2018) ; Conférencier invité au département de la santé et des soins, municipalité d'Aarhus, Danemark (2015).

\subsection{Affiliations professionnelles}
\vspace{-.25em}
\begin{itemize}[itemsep=0em, topsep=0em, partopsep=0em]
	\item Université de Loughborough Centre for Research in Communication and Culture (Political Communication Theme), Responsable (2024-)
	\item UK Young Academy, Membre (2023-2027)
	\item British Academy Early Career Researcher Network, Membre (2022-)
	\item Higher Education Academy, « Fellow » (2021-)
	\item Université de Loughborough Centre for Research in Communication and Culture (Political Communication Theme), Membre (2019-)
	\item Political Institutions, Behaviour and Decision Making Research Group, Membre (2017-2019)
	\item Réseau d’études sur les politiques d'inégalité, Membre (2015-2017)
	\item Groupe de recherche sur l'universalisme, Membre (2014-2017)
	\item Centre pour l’étude de la citoyenneté démocratique, Membre étudiant (2013-2014)
	\item Institut universitaire européen, Chercheur invité (2013)\\
\end{itemize}

\phantomsection
\addcontentsline{toc}{subsection}{Commentateur de panel}\textit{Commentateur de panel} : Conférence annuelle de l'Association néerlandaise-flamande de science politique (2018); Conférence annuelle de la CES (2017) ; Conférence annuelle de l'Association danoise de science politique (2015).\\ 
\hfill \break
\phantomsection
\addcontentsline{toc}{subsection}{Président de panel}\textit{Président de panel} : Elections, Public Opinion and Parties Conférence (2024); Conférence annuelle de l'Association européenne de science politique (2019; 2023).\\ 
\hfill \break
\phantomsection
\addcontentsline{toc}{subsection}{Comité de lecteurs}\textit{Comité de lecteurs} : Bloomsbury I.B. Tauris; Economic and Social Research Council; Oxford University Press; Conseil de recherches en sciences humaines du Canada; \textit{Administration \& Society; Acta Sociologica; American Political Science Review; British Journal of Political Science; Canadian Journal of Political Science; Canadian Public Policy; Comparative Political Studies; Economics \& Politics; European Journal of Political Research; European Journal of Social Security; European Political Science Review; European Sociological Review; Governance; International Journal of Comparative Sociology; International Journal of Public Opinion Research; International Journal of Social Welfare; International Political Science Review; Journal of European Social Policy; Journal of Social Policy; Journal of Politics; Journal of Public Policy; Party Politics; Policy Sciences; Political Behavior; Political Studies; Political Studies Review; Public Performance \& Management Review; Social Policy \& Administration; Social Policy and Society; Socio-Economic Review}.

%LANGUAGES
\section{Langues}
\begin{itemize}[itemsep=0em, topsep=0em, partopsep=0em]
	\item Anglais : langue maternelle
	\item Français : C1 -- niveau autonome
	\item Italien : B2 -- niveau avancé
	\item Danois : B1 -- niveau seuil \\
\end{itemize}

%CITIZENSHIP
\phantomsection
\addcontentsline{toc}{section}{Citoyenneté}\textbf{Citoyenneté} : Canadienne\\

% %REFERENCES
% \vspace{-1em}
% \section{Références}

% \begin{minipage}[t]{0.54\textwidth}
% 	Carsten Jensen, Professeur titulaire\\
% 	Department of Political Science\\
% 	Aarhus University\\
% 	Bartholins Allé 7, Building 1340, Rm 227\\
% 	8000 Aarhus C, Denmark\\
% 	Téléphone : +45 87165678\\
% 	Courriel : carstenj@ps.au.dk\\
% \end{minipage}
% \begin{minipage}[t]{0.46\textwidth}
% 	Kees van Kersbergen, Professeur titulaire\\
% 	Department of Political Science\\
% 	Aarhus University\\
% 	Bartholins Allé 7, Building 1340, Rm 229\\
% 	8000 Aarhus C, Denmark\\
% 	Téléphone : +45 87165727\\
% 	Courriel : kvk@ps.au.dk\\
% \end{minipage}

% \bigskip
% \begin{minipage}[t]{0.54\textwidth}
% 	Stuart Soroka, Professeur titulaire\\
% 	Communication Studies \& Political Science\\
% 	University of Michigan\\
% 	Room 4448, 426 Thompson Street\\
% 	Ann Arbor, MI 48104–2321\\
% 	Téléphone : +1 734 764 8362\\
% 	Courriel : ssoroka@umich.edu\\
% \end{minipage}
% \begin{minipage}[t]{0.46\textwidth}
% 	Barbara Vis, Professeur titulaire\\
% 	School of Governance\\
% 	Utrecht University\\
% 	Bijlhouwerstraat 6\\
% 	3511 ZC Utrecht, The Netherlands\\
% 	Téléphone : +31(0)30 253 4935\\
% 	Courriel : b.vis@uu.nl
% \end{minipage}

\end{document}